\documentclass[10pt,a4paper]{letter}
\usepackage[latin1]{inputenc}
\usepackage{utopia}
\usepackage{amsmath}
\usepackage{amsfonts}
\usepackage{amssymb}
\usepackage{graphicx}
\usepackage{float}
\usepackage{url}
\usepackage{hyperref}
\usepackage[left=1in, right=1in, bottom=1in, top=1in]{geometry}
\renewcommand{\baselinestretch}{1.2}

\usepackage[table,xcdraw]{xcolor}
\usepackage[dvips]{xcolor}
\definecolorseries

%\usepackage{notomath}
%\usepackage[T1]{fontenc}


\hypersetup{colorlinks,linkcolor={[cmyk]0,0.88502,1,0},citecolor={[cmyk]0,0.88502,1,0}, filecolor={violet}, urlcolor={[cmyk]0,0.88502,1,0}}  

\begin{document}
	
{\Large 
	\textbf{Rising Inflation Threatens to Swamp Cambodian Households}
} 

The Diplomat, May 26, 2021 \\
By Nith Kosal

\textit{Many Cambodian households were in a precarious position even before COVID-19 hit.
}

After the end of the financial crisis in 2009, Cambodia's economy experienced vigorous growth, with GDP growth rates reaching 7 percent or more each year. But that growth has dropped off dramatically as a consequence of the COVID-19 pandemic. Hundreds of thousands of workers in the garment and footwear manufacturing sector, as well as in transport, tourism, and construction, have been affected by a sharp recession that has brought widespread job and income losses.

The shock has been caused in part by steep drops in household consumption expenditures, household savings, new FDI investment, and the amount and range of products being exported. Cambodia's economic output shrunk by an unprecedented 3.1 percent in 2020 as a result of the pandemic. At the time, the Consumer Price Index (CPI) for December had reflected a year-over-year \href{https://www.nis.gov.kh/nis/cpi/2020/PP_CPI%20summary%20table%20Dec%202020.htm}{upturn} of 2.9 percent, which at a glance appears to be an uncomfortably high rate of inflation.

At this point in the pandemic, Cambodia --- like much of the world --- is in a position of \href{https://www.ineteconomics.org/uploads/papers/WP_92-Frydman-et-al-KUH.pdf}{Knightian uncertainty}, in which we do not know the outcome of a given situation and cannot know all the information we need to make accurate assessments about the future. When we talk about inflation and interest rates, we are doing so in a context where there are many things that we do not know and which will likely remain unknown for a very long time.

There is uncertainty about the \href{https://www.econlib.org/library/Enc/PhillipsCurve.html}{Phillips curve}, the real unemployment rate and its relationship to the inflation rate. There is also uncertainty about multipliers, uncertainty about central bank dovishness, uncertainty about how much of the budget will support vulnerable people, and uncertainty about the size and the financing of the government's economic stimulus package. All these uncertainties have a negative impact on governments' and economists' ability to produce policy recommendations aimed at solving the problem.

Despite all of the uncertainties, however, rising inflation is a problem that Cambodian economists and relevant government officials must be focused on. The issue looms as a grave concern, especially for low-income households, the unemployed, and those who make their living in the informal economy.

Inflation, or the rate of change in prices for a basket of goods and services, changes over time; it's not a simple phenomenon to measure and interpret. Inflation that is perpetually too high can hurt household welfare, particularly when offset by a comparable increase in incomes, which creates a vicious cycle. When inflation is persistently too low, however, it can be difficult for the central bank to adjust monetary policy by setting interest and exchange rates to sustain the economy. It is a sign the economy is below its capacity. It can also negatively affect certain investment decisions.

The International Monetary Fund (IMF) has projected that Cambodia's inflation \href{https://www.imf.org/-/media/Files/Publications/WEO/2021/April/English/text.ashx}{growth} rate to be 3.1 percent in 2021 and 2.8 percent in 2022. While we do not yet have monthly price data from January to April of this year, we can anticipate accelerated inflation that will pose challenges for policymakers, investors, and family economies in the year to come.

In April 2020, a time of relative economic standstill in Phnom Penh and surrounding areas as COVID-19 began to hit, the CPI was \href{https://www.nis.gov.kh/nis/cpi/2020/PP_CPI%20summary%20table%20Apr%202020.htm}{1.9 percent higher} compared to the same month in 2019. For instance, ``core" CPI items such as food and non-alcoholic beverages represented a large change of 4.4 percent, while the price of housing, water, electricity, gas, and other fuels increased by only 0.1 percent due to lower global oil prices.

Twelve months later, in April of this year, Phnom Penh and nearby Takhmau city were placed under lockdown with strict travel restrictions. On April 14, an audio message from Prime Minister Hun Sen about his decision to close the city was \href{https://www.phnompenhpost.com/national/pm-officials-be-punished-leaking-audio-lockdown-order}{leaked} before the official announcement. In the absence of immediate clarification by the government, the leak prompted many citizens to rush out and buy food, causing a sharp rise in the price of some goods.

Even before this strict lockdown, starting from late February some essential daily commodities were already significantly increasing in price, according to \href{https://www.khmertimeskh.com/710643/pandemic-causes-rise-in-consumer-price-index/}{a report} from the Ministry of Commerce. The report showed that rice prices in Siem Reap and Preah Sihanouk provinces rose as dramatically as 33.33 percent and 17.56 percent, respectively. At the same time, in Kandal, Svay Rieng, Kampong Cham, and Battambang provinces, the price of beef rose by about 10 percent and pork prices increased by as much as 59 percent. In addition to food, basic medical goods also increased in price. The prices of surgical masks, hand sanitizers, and temperature kits, in high demand at the time, spiked from 30 percent to 100 percent, depending on the region.

Compounding the problem is the fact that Cambodian households have clearly experienced a decrease in their purchasing power. The World Bank recently released a \href{http://documents1.worldbank.org/curated/en/480321617235189070/pdf/Main-Report.pdf}{fourth-round} result of its High-Frequency Phone Survey of Households (HFPS) in Cambodia. HFPS is a follow-up interview by telephone with more than 1,680 respondents, including 1,277 representative households with an IDPoor equity card --- which grants beneficiaries access to services like subsidized schooling and free health care, as well as making them eligible for the government's COVID-19 cash transfer program --- and 410 representative households without an IDPoor equity card. The result revealed that as of December, 32 percent for those households with IDPoor cards experienced moderate to severe food insecurity. The same was true of 17 percent of those families without IDPoor cards.

Also from the HFPS results, 10 percent of respondents reported having lost their jobs compared to the pre-pandemic period in January 2020. The report also noted that nearly one in two households reported a loss in income: 48 percent of the households without an ID Poor and 46 percent of households with an ID Poor reported a decline in household income in December.

At the same time, \href{http://www.angkorresearch.com/imgs/file/The%20Effect%20of%20Covid-19%20Impacts%20on%20Cambodian%20Development_%20Headline%20Results%20Brief_R3_EN_V4.pdf}{a third-round} result of a COVID-19 Economic Impact Survey conducted by Angkor Research and Consulting, in partnership with Future Forum, found that 12 percent of households reported a reduction in their savings between January to October 2020 and families with any type of loan increased their borrowing by 10 percent over the same period. It appears that loans are mainly used for daily expenses such as food, to repay other loans, and to pay for healthcare services.

Indeed, starting from the first positive case of the virus in January of last year and the resulting restrictions imposed to deal with the pandemic, demand for many services has dramatically reduced. This phenomenon may have prompted restaurants, hotels, and travel agencies that remained open to raise their prices. Macroeconomists call inflation caused by such surges in spending ``demand-pull inflation."

On the other side, several inputs into agriculture and other industries have increased the cost of products because of the lack of production and some issues in value chains and supply chains. Economists categorize this type of inflation as ``cost-pull inflation."

The overarching policy measures undertaken by the Cambodian government to control the pandemic have not focused on the problem of rising prices, particularly for food products. The relevant authorities must turn their attention to this problem and to the bigger issue of inflation, with an emphasis on rapid reaction and reform of monetary policy in response to any transitory rise in prices. If they can do that, of course, it will benefit Cambodian household consumption overall.

Given the precarious situation of many Cambodian households, the administration needs to carefully monitor both actual price changes and inflation expectations for any signs of unexpected price pressures that might arise even Cambodia leaves the pandemic behind and enters the next economic expansion.

\textit{Nith Kosal is a junior research fellow at Future Forum, an independent public policy think tank based in Phnom Penh. His research interests include applied economics, macroeconomics, economic development and international economics.}
\end{document}