\documentclass[11pt,a4paper]{article}
\usepackage[latin1]{inputenc}
\usepackage[T1]{fontenc}
\usepackage{amsmath}
\usepackage{amsfonts}
\usepackage{amssymb}
\usepackage{graphicx}
\usepackage[width=16.00cm, height=24.00cm]{geometry}
\author{Nith Kosal}
\renewcommand{\baselinestretch}{1}
\begin{document}
	\begin{center}
		\large \textbf{Short Biography}
	\end{center}
	
	Nith Kosal is a Junior Research Fellow and Data Hub Coordinator at Future Forum, an independent think tank generating new thinking for a new Cambodia. He is a co-founder and a young macroeconomist of the Sethakech, an online community for sharing and discussion about economics in theory and practice in the real world. Kosal is a former young research fellow at Future Forum in 2020 working under a research grant on government ownership of banks.
	
	In 2018, he was co-author of Reinvigorating Cambodian Agriculture, which won the Best Paper Award at the 5th National Bank of Cambodia Annual Macroeconomic Conference. He also won first place in the Essay Contest 2019the 4th Industrial Revolution, organized by the Cambodia Development Center. Professionally, Kosal has produced several numbers of op-ed in different media outlets such as The Diplomat, The Phnom Penh Post, East Asia Forum, Southeast Asia Globe, and VOD. 
			
	Outside the academic background, Kosal was a student representative and general coordinator in the French Department of Economics and Management. In his third year, he completed a three-month internship under the Youth Resource Development Program (YRDP) as a program assistant working in support of project planning and coordination of activities and research for good governance in the extractive industries. In 2017, Kosal co-founded the Chea Sim Reak Chey High School Alumni Association and also volunteered in market research with the YRDP and the Koonsoor Kampuchea Training Academy. At the same time, has volunteered as a Country Leadership Trainer for Habitat for Humanity, Cambodia since 2018. 
	
	He holds a double bachelor's degree in Economics from Universit� Lumi�re Lyon 2 (France) and the Royal University of Law and Economics (Cambodia) in July 2019, which is the 25th generation. His research interests include applied economics, macroeconomics, economic development, and international economics. Kosal has participated in numerous social work projects through various study visits and advocacy activities to promote the protection of natural resources and sustainable development in Cambodia. 
	
	
	\flushright This short biography was last fully updated on \today.
\end{document}